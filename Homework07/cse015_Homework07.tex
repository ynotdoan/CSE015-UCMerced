\documentclass[11pt]{article}

% To produce a letter size output. Otherwise will be A4 size.
\usepackage[letterpaper]{geometry}

% For enumerated lists using letters: a. b. etc.
\usepackage{enumitem}

\topmargin -.5in
\textheight 9in
\oddsidemargin -.25in
\evensidemargin -.25in
\textwidth 7in

\begin{document}

% Edit the following putting your first and last names and your lab section.
\author{Tony Doan\\
Lab CSE-015-11L }

% Edit the following replacing X with the HW number.
\title{CSE 015: Discrete Mathematics\\
Fall 2020\\
Homework \#07\\
Solution}

% If you put nothing today's date will be included. Alternatively, uncomment the next line and put the date you want to appear
\date{November 28, 2020}
\maketitle

% ========== Begin questions here
\begin{enumerate}
\item
\textbf{Question 1: Asymptotic Notation} \\ \\ % Q1 begins
g(n) = O($n^{2}$)
\begin{enumerate}[label=(\alph*)]
\item % 1a
f(n) = 178n + 45 \\ 
The largest term is 178n \\
Using the rule for O(n) that states f(n) must be less than or equal to g(n) [f(n) $\leq$ g(n)],
we see in this problem that O(n) $\leq$ O($n^{2}$) is true. \\
Therefore, the answer would be yes, because n increases slower than $n^{2}$. \\
\item % 1b
f(n) = nlogn + 12 \\
The largest term is nlogn \\
Using the rule for O(n) that states f(n) must be less than or equal to g(n) [f(n) $\leq$ g(n)],
we see in this problem that O(logn) $\leq$ O($n^{2}$) is true. \\
Therefore, the answer would be yes, because logn increases slower than $n^{2}$. \\
\item % 1c
f(n) = 34$n^{2}$ + 34n + 34 \\
The largest term is 34$n^{2}$ \\
Using the rule for O(n) that states f(n) must be less than or equal to g(n) [f(n) $\leq$ g(n)],
we see in this problem that O($n^{2}$) $\leq$ O($n^{2}$) is true. \\
Therefore, the answer would be yes, because $n^{2}$ increases at the same rate as $n^{2}$. \\
\item % 1d
f(n) = $\sqrt{n}$ + 2 \\
The largest term is $\sqrt{n}$, or $n^{\frac{1}{2}}$ \\
Using the rule for O(n) that states f(n) must be less than or equal to g(n) [f(n) $\leq$ g(n)],
we see in this problem that O($n^{\frac{1}{2}}$) $\leq$ O($n^{2}$) is true. \\
Therefore, the answer would be yes, because $n^{\frac{1}{2}}$ increases slower than $n^{2}$. \\
\item % 1e
f(n) = 0.001$n^{3}$ + 72n \\
The largest term is 0.001$n^{3}$ \\
Using the rule for O(n) that states f(n) must be less than or equal to g(n) [f(n) $\leq$ g(n)],
we see in this problem that O($n^{3}$) $\leq$ O($n^{2}$) is false. \\
Therefore, the answer would be no, because $n^{3}$ increases faster than $n^{2}$. \\
\end{enumerate}
% Q1 end

\item
\textbf{Question 2: Asymptotic Notation} \\ \\ % Q2 begins
In order from the top being number 1, or increasing fastest, to bottom being number 9, or increasing slowest. 
\begin{itemize}
\item % 1
logn 
\item % 2
$\sqrt{n}$ 
\item % 3
nlogn 
\item % 4
n
\item % 5
$n^{2}$logn 
\item % 6
$n^{2}$ 
\item % 7
$n^{4}$ 
\item % 8
$2^{n}$ 
\item % 9
$3^{n}$ 
\end{itemize}
% Q2 end

\item
\textbf{Question 3: Asymptotic Growth} % Q3 begins
\begin{itemize}
\item % time complexities
$f_{1}$(n) = 5$n^{2}$ + 34n + 12
\item
$f_{2}$(n) = 10n + 4
\item 
$f_{3}$(n) = $2^{n}$
\\ \\
\textbf{Computer A: } \\
Converting $10^{6}$ operations per second to operations per hour: $10^{9}$ \\
\\
$f_{1}$(n) = 5$n^{2}$ + 34n + 12 $\leq$ 3.6 * $10^{9}$ \\
Solving for n: \\
n = 26829.42 \\
Computer A can perform 26829.42 operations per hour for $f_{1}$. \\
\\
$f_{2}$(n) = 10n + 4 $\leq$ 3.6 * $10^{9}$ \\
Solving for n: \\
n = 359999999.6 \\
Computer A can perform 359999999.6 operations per hour for $f_{2}$. \\
\\
$f_{3}$(n) = $2^{n}$ $\leq$ 3.6 * $10^{9}$ \\
Solving for n: \\
n = 31.75 \\
Computer A can perform 31.75 operations per hour for $f_{3}$. \\
\\
\textbf{Computer B: } \\
Converting $10^{8}$ operations per second to operations per hour: $10^{11}$ \\
\\
$f_{1}$(n) = 5$n^{2}$ + 34n + 12 $\leq$ 3.6 * $10^{11}$ \\
Solving for n: \\
n = 268324.76 \\
Computer B can perform 268324.76 operations per hour for $f_{1}$. \\
\\
$f_{2}$(n) = 10n + 4 $\leq$ 3.6 * $10^{11}$ \\
Solving for n: \\
n = 35999999999.6 \\
Computer B can perform 35999999999.6 operations per hour for $f_{2}$. \\
\\
$f_{3}$(n) = $2^{n}$ $\leq$ 3.6 * $10^{11}$ \\
Solving for n: \\
n = 38.39 \\
Computer B can perform 38.39 operations per hour for $f_{3}$. \\
\\
\end{itemize}
% Q3 end
\end{enumerate}
\end{document}
