\documentclass[11pt]{article}

% To produce a letter size output. Otherwise will be A4 size.
\usepackage[letterpaper]{geometry}

% For enumerated lists using letters: a. b. etc.
\usepackage{enumitem}

\topmargin -.5in
\textheight 9in
\oddsidemargin -.25in
\evensidemargin -.25in
\textwidth 7in

\begin{document}

% Edit the following putting your first and last names and your lab section.
\author{Tony Doan\\
Lab CSE-015-11L }

% Edit the following replacing X with the HW number.
\title{CSE 015: Discrete Mathematics\\
Fall 2020\\
Homework \#4\\
Solution}

% If you put nothing today's date will be included. Alternatively, uncomment the next line and put the date you want to appear
\date{October 18, 2020}
\maketitle

% ========== Begin questions here
\begin{enumerate}
\item
\textbf{Question 1: Set Operations} % Q1 begins
\begin{enumerate}[label=(\alph*)]
\item % 1a
A $\cup$ B \\
UCM students are either registered in CSE015 or live in Merced county.

\item % 1b
A $\cap$ C \\
UCM students are both registered in CSE015 and freshmen.

\item % 1c
C $\setminus$ B \\
UCM students who live in Merced county but who are not freshmen.

\item % 1d
$\overline {A}$ \\
UCM students who are not registered in CSE015.

\item % 1e
A $\cap$ B $\cap$ C \\
UCM students who are registered in CSE015, live in Merced county, and are freshmen.
\end{enumerate}
% Q1 end

\item
\textbf{Question 2: Cartesian Product } % Q2 begins
\begin{enumerate}[label=(\alph*)]
\item % 2a
C $\times$ A \\
\textit {
\{("True", 1), ("True", 2), ("True", 3), ("True", 4), ("False", 1), ("False", 2), ("False", 3), ("False", 4)\}
}

\item % 2b
B $\times$ B \\
\textit {
\{("a", "a"), ("a", "b"), ("a", "c"), ("b", "a"), ("b", "b"), ("b", "c"), ("c", "a"), ("c", "b"), ("c", "c")\}
}

\item % 2c
B $\times$ A $\times$ C \\
\textit {
\{("a", 1, "True"), ("a", 1, "False"), ("a", 2, "True"), ("a", 2, "False"), ("a", 3, "True"), ("a", 3, "False"), ("a", 4, "True"), ("a", 4, "False"), ("b", 1, "True"), ("b", 1, "True"), ("b", 2, "True"), ("b", 2, "False"), ("b", 3, "True"), ("b", 3, "False"), ("b", 4, "True"), ("b", 4, "False"), ("c", 1, "True"), ("c", 1, "True"), ("c", 2, "True"), ("c", 2, "False"), ("c", 3, "True"), ("c", 3, "False"), ("c", 4, "True"), ("c", 4, "False")\}
}
\end{enumerate}
% Q2 end

\item
\textbf {Question 3: Composite Cartesian Products } % Q3 begins
\begin {enumerate} 
A $\times$ (B $\cup$ C) $\equiv$ (A $\times$ B) $\cup$ (A $\times$ C) \\
\underline {Proof:} \\
= A $\times$ (B $\cup$ C) = \{(x, y) $\vline$ x $\in$ A and y $\in$ B $\cup$ C\} \\
= \{(x, y) $\vline$ ((x, y) $\in$ A $\times$ B) $\cup$ ((x, y) $\in$ A $\times$ C)\} \\
= (A $\times$ B) $\cup$ (A $\times$ C) \\
\newline
True, because if x $\in$ A and y $\in$ B $\cup$ C, there will be (x, y) for either or both A $\times$ B and A $\times$ C.
\end {enumerate}
% Q3 end

\item
\textbf{Question 4: Relations } % Q4 begins
\begin{enumerate}[label=(\alph*)]
\item % 4a
\textit {
$R_{1}$ = \{(a ,b), (a, c), (a, a), (b, a), (c, a)\} \\
}
\begin {tabular} {c|c}
    \textbf {1} & \textbf {2} \\
    \hline
    a & b \\
    a & c \\
    a & a \\
    b & a \\
    c & a \\
\end {tabular} \\
\newline
\textbf {
Reflexive:
} \\
\textit {
(a, a) but no (b, b), (c, c), or (d, d) \\
}

\textbf {
Symmetric: 
} \\
\textit {
(a, b), (b, a) \\
(a, c), (c, a) \\
(a, a) \\
}

\textbf {
Anti-Symmetric: 
} \\
\textit {
(a, b), (b, a) \\
(a, c), (c, a) \\
(a, a) \\
If (a, b) were to be swapped to (b, a), it would cause there to be two pairs (b, a), so it can't be anti-symmetric. \\
}

\textbf {
Transitive: 
} \\
\textit {
(a, b) to (b, a) to (a, a) \\
(b, a) to (a, b) but no (b, b) \\
(c, a) to (a, c) \\
}

It would be symmetric. \\

\item % 4b
\textit {
$R_{2}$ = \{(a, b), (b, b), (b, c), (c, c), (a, c)\} \\
}
\begin {tabular} {c|c}
    \textbf {1} & \textbf {2} \\
    \hline
    a & b \\
    b & b \\
    b & c \\
    c & c \\
    a & c \\
\end {tabular} \\
\newline
\textbf {
Reflexive: 
} \\
\textit {
There is a (b, b), (c, c) but not (a, a) or (d, d).
}

\textbf {
Symmetric: 
} \\
\textit {
(a, b) but no (b, a) \\
(b, c) but no (c, b) \\
(a, c) but no (c, a) \\
}

\textbf {
Anti-Symmetric: 
} \\
\textit {
The above explanation for symmetric shows that it can be anti-symmetric because if we flip the elements, there are no pairs that match. \\
}

\textbf {
Transitive: 
} \\
\textit {
(a, b) to (b, b) to (a, b) \\
(a, b) to (b, c) to (a, c) \\
(b, c) to (c, c) to (b, c) \\
}

It would be both transitive and anti-symmetric. \\

\item % 4c
\textit {
$R_{3}$ = \{(a, b), (d, c), (c, a), (c, d), (a, b)\} \\
}
\begin {tabular} {c|c}
    \textbf {1} & \textbf {2} \\
    \hline
    a & b \\
    d & c \\
    c & a \\
    c & d \\
    a & b \\
\end {tabular} \\
\newline
\textbf {
Reflexive: 
} \\
\textit {
No (a, a), (b, b), (c, c), (d, d) \\
There are no reflexive pairs. \\
}

\textbf {
Symmetric: 
} \\
\textit {
(a, b) but no (b, a) \\
(d, c) and (c, d) \\
(c, a) but no (a, c) \\
(a, b) repeats \\
}

\textbf {
Anti-Symmetric: 
} \\
\textit {
(a, b) repeats \\
}

\textbf {
Transitive: 
} \\
\textit {
(a, b) but no other elements with b \\
(d, c) to (c, d) \\
(c, a) to (a, b) but no (b, c) \\
}

It would be none of the above because it doesn't fit the criteria. \\

\item % 4d
\textit {
$R_{4}$ = \{(a, a), (b, b), (c, c)\} \\
}
\begin {tabular} {c|c}
    \textbf {1} & \textbf {2} \\
    \hline 
    a & a \\
    b & b \\
    c & c\\
\end {tabular} \\
\newline
\textbf {
Reflexive: 
} \\
\textit {
There is (a, a), (b, b), (c, c) but no (d, d) \\
}

\textbf {
Symmetric: 
} \\
\textit {
All the elements could be swapped and still be the same with no repeats. \\
}

\textbf {
Anti-Symmetric: 
} \\
\textit {
All the elements could be swapped and still be the same with no repeats. \\
}

\textbf {
Transitive: 
} \\
\textit {
None of them equal each other and transitive can not have an element point to itself, like (a, a). \\
}

It would be both symmetric and anti-symmetric. \\
\end {enumerate}
% Q4 end

\item 
\textbf {Question 5: Functions } % Q5 begins
\begin {enumerate} [label = (\alph*)] 
\item % 5a
\textit {
f(m, n) = 2m - n \\
}
Surjective, because you set m to 0 and n to any integer, you get (0, n). If you plug in any integer for n, you will get an output no matter what you choose. \\

\item % 5b
\textit {
f(m, n) = $m^{2}$ - $n^{2}$ \\
}
Not surjective, because it is a perfect square so (m, n) can not equal 2. It can only equal to 1 or greater than 2. \\
\textit {
f(1, 0) = $1^{2}$ - $0^{2}$ = 1 - 0 = 1 \\
f(2, 0) = $2^{2}$ - $0^{2}$ = 4 - 0 = 4 \\
}

\item % 5c
\textit {
f(m, n) = $\left|m\right|$ - $\left|n\right|$ \\
}
Surjective, because any integer that is plugged in will have an output of all integers. \\

\item % 5d
\textit {
f(m, n) = $m^{2}$ - 4 \\
}
Not surjective, because you can not get all the integers. The lowest integer that you can get is -4. \\
\textit {
f(0, n) = $0^{2}$ - 4 = 0 - 4 = -4 \\
f(-1, n) = $-1^{2}$ - 4 = 1 - 4 = -3 \\
f(1, n) = $1^{2}$ - 4 = 1 - 4 = -3 \\
}
\end {enumerate}
% Q5 end
\end{enumerate}
\end{document}
