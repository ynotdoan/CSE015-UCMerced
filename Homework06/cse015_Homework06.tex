\documentclass[11pt]{article}

% To produce a letter size output. Otherwise will be A4 size.
\usepackage[letterpaper]{geometry}

% For enumerated lists using letters: a. b. etc.
\usepackage{enumitem}

\topmargin -.5in
\textheight 9in
\oddsidemargin -.25in
\evensidemargin -.25in
\textwidth 7in

\begin{document}

% Edit the following putting your first and last names and your lab section.
\author{Tony Doan \\
Lab CSE-015-11L }

% Edit the following replacing X with the HW number.
\title{CSE 015: Discrete Mathematics\\
Fall 2020\\
Homework \#6\\
Solution}

% If you put nothing today's date will be included. Alternatively, uncomment the next line and put the date you want to appear
\date{November 14, 2020 }
\maketitle

% ========== Begin questions here
\begin{enumerate}
\item
\textbf{Question 1: Recursively defined functions } \\ % Q1 begins
f(0) = 3 for n = 1, 2, 3, ...
\begin{enumerate}[label=(\alph*)]
\item % 1a
f(n + 1) = -2f(n) \\
f(1) = f(0 + 1) = -2f(0) = -2 * 3 = -6 \\
f(2) = f(1 + 1) = -2f(1) = -2 * -6 = 12 \\
f(3) = f(2 + 1) = -2f(2) = -2 * 12 = -24 \\
f(4) = f(3 + 1) = -2f(3) = -2 * -24 = 48 \\
f(5) = f(4 + 1) = -2f(4) = -2 * 42 = -96 
\item % 1b
f(n + 1) = 3f(n) + 7 \\
f(1) = f(0 + 1) = 3f(0) + 7 = (3 * 3) + 7 = 16 \\
f(2) = f(1 + 1) = 3f(1) + 7 = (3 * 16) + 7 = 55 \\
f(3) = f(2 + 1) = 3f(2) + 7 = (3 * 55) + 7 = 172 \\
f(4) = f(3 + 1) = 3f(3) + 7 = (3 * 172) + 7 = 523 \\
f(5) = f(4 + 1) = 3f(4) + 7 = (3 * 523) + 7 = 1576 
\item % 1c
f(n + 1) = $f(n)^2$ - 2f(n) - 2 \\
f(1) = f(0 + 1) = $f(0)^2$ - 2f(0) - 2 = $3^2$ - (2 * 3) - 2 = 9 - 6 - 2 = 1 \\
f(2) = f(1 + 1) = $f(1)^2$ - 2f(1) - 2 = $1^2$ - (2 * 1) - 2 = 1 - 2 - 2 = -3 \\
f(3) = f(2 + 1) = $f(2)^2$ - 2f(2) - 2 = $-3^2$ - (2 * -3) - 2 = 9 - (-6) - 2 = 13 \\
f(4) = f(3 + 1) = $f(3)^2$ - 2f(3) - 2 = $13^2$ - (2 * 13) - 2 = 169 - 26 - 2 = 141 \\
f(5) = f(4 + 1) = $f(4)^2$ - 2f(4) - 2 = $141^2$ - (2 * 141) - 2 = 19881 - 282 - 2 = 19597 
\item % 1d
f(n + 1) = $3^{(\frac{f(n)}{3})}$ \\ 
f(1) = f(0 + 1) = $3^{(\frac{f(0)}{3})}$ = $3^{(\frac{3}{3})}$ = $3^{(1)}$ = 3 \\ 
f(2) = f(1 + 1) = $3^{(\frac{f(1)}{3})}$ = $3^{(\frac{3}{3})}$ = $3^{(1)}$ = 3 \\ 
Since f(1) = 3 and f(2) = 3, plugging in 3 for f(3), f(4), f(5) will also equal to 3 since they use the same equation recursively... \\
f(3) = 3 \\
f(4) = 3 \\
f(5) = 3 
\\
\end{enumerate}
% Q1 ends

\item
\textbf{Question 2: Recursively Defined Sequences } % Q2 begins
\begin{enumerate}[label=(\alph*)]
\item % 2a
$a_n$ = 4n - 2 \\
\\
$a_0$ = 4(0) - 2 = -2 \\
$a_1$ = 4(1) - 2 = 2 \\
$a_2$ = 4(2) - 2 = 6 \\
$a_3$ = 4(3) - 2 = 10 \\
\\
Basis: \\
$a_0$ = 4(0) - 2 = -2 \\
\\
Induction: \\
Assume n = k... $a_k$ = 4k - 2 \\
Let n = k + 1... $a_{k+1}$ = 4(k + 1) - 2 \\
$a_{k+1}$ = 4(k + 1) - 2 \\
$a_{k+1}$ = 4k + 4 - 2 \\
$a_{k+1}$ = 4k + 2, True \\

\item % 2b
$a_n$ = 1 + $(-1)^n$ \\
\\
$a_0$ = 1 + $(-1)^0$ = 1 + 1 = 2 \\
$a_1$ = 1 + $(-1)^1$ = 1 + (-1) = 0 \\
$a_2$ = 1 + $(-1)^2$ = 1 + 1 = 2 \\
$a_3$ = 1 + $(-1)^3$ = 1 + (-1) = 0 \\
\\
Basis: \\
$a_0$ = 1 + $(-1)^0$ = 1 + 1 = 2 \\
\\
Induction: \\
Assume n = k... $a_k$ = 1 + $(-1)^k$ \\
Let n = k + 1... $a_{k + 1}$ = 1 + $(-1)^{k + 1}$ \\
$a_{k + 1}$ = 1 + $(-1)^{k + 1}$ \\
$a_{k + 1}$ = 1 + $(-1)^{k + 1}$, True \\

\item % 2c
$a_n$ = n(n - 1) \\
\\
$a_0$ = 0(0 - 1) = 0 * -1 = 0 \\
$a_1$ = 1(1 - 1) = 1 * 0 = 0 \\
$a_2$ = 2(2 - 1) = 2 * 1 = 2 \\
$a_3$ = 3(3 - 1) = 3 * 2 = 6 \\
\\
Basis: \\
$a_0$ = 0(0 - 1) = 0 * -1 = 0 \\
\\ 
Induction: \\
Assume n = k... $a_k$ = k(k - 1) \\
Let n = k + 1... $a_{k + 1}$ = (k + 1) ((k + 1) - 1) \\
$a_{k + 1}$ = (k + 1) ((k + 1) - 1) \\
$a_{k + 1}$ = (k + 1) (k) \\
$a_{k + 1}$ = $k^2$ + k, True \\

\item % 2d
$a_n$ = $n^2$ \\
\\
$a_0$ = $0^2$ = 0 \\
$a_1$ = $1^2$ = 1 \\
$a_2$ = $2^2$ = 4 \\
$a_3$ = $3^2$ = 9 \\
\\
Basis: \\
$a_0$ = $0^2$ = 0 \\
\\ 
Induction: \\
Assume n = k... $a_k$ = $k^2$ \\
Let n = K + 1... $a_{k + 1}$ = $(k + 1)^2$ \\
$a_{k + 1}$ = $(k + 1)^2$ \\
$a_{k + 1}$ = $k^2$ + 2k + 1, True \\
\end{enumerate}
% Q2 ends

\item
\textbf{Question 3: Recursively Defined Sets } \\ % Q3 begins
$\Sigma$ = {0, 1} \\
S = \{(0, 1), (00, 11), (000, 111), (0000, 1111), (00000, 11111), ... \} \\
In set S for whatever number of "0" there is, it combined with the equal amount of "1". \\
% Q3 end
\end{enumerate}
\end{document}
