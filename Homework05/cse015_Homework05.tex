\documentclass[11pt]{article}

% To produce a letter size output. Otherwise will be A4 size.
\usepackage[letterpaper]{geometry}

% For enumerated lists using letters: a. b. etc.
\usepackage{enumitem}

\usepackage{amsmath}
\usepackage[makeroom]{cancel}

\topmargin -.5in
\textheight 9in
\oddsidemargin -.25in
\evensidemargin -.25in
\textwidth 7in

\begin{document}

% Edit the following putting your first and last names and your lab section.
\author{Tony Doan\\
Lab CSE-015-11L }

% Edit the following replacing X with the HW number.
\title{CSE 015: Discrete Mathematics\\
Fall 2020\\
Homework \#5\\
Solution}

% If you put nothing today's date will be included. Alternatively, uncomment the next line and put the date you want to appear
\date{November 1, 2020}
\maketitle

% ========== Begin questions here
\begin{enumerate} 
\item
\textbf{Question 1: Mathematical Induction 1} \\ \\ % Q1 begins
P(n) = 1$^{3}$ + 2$^{3}$ + 3$^{3}$ + ... + n$^{3}$ = $(\frac {n(n + 1)}{2})^{2}$
\begin{enumerate}[label=(\alph*)]
\item % 1a
\textbf {
P(1): \\
}
1$^{3}$ = $(\frac {1(1 + 1)}{2})^{2}$ \\ 

\item % 1b
1$^{3}$ = $(\frac {1(1 + 1)}{2})^{2}$ \\
1 = $(\frac {1(2)}{2})^{2}$ \\ 
1 = $(\frac {1 \cancel{(2)}} {\cancel{2}})^{2}$ \\
1 = $1^{2}$ \\
1 = 1, True \\

\item % 1c
Assume n = k is true. \\
\textbf {
P(k): \\
}
k$^{3}$ = $(\frac {k(k + 1)}{2})^{2}$ \\

\item % 1d
Add (k + 1)$^{3}$ to both sides \\
\textbf {
P(k + 1): \\
}
k$^{3}$ + (k + 1)$^{3}$ = $(\frac {k(k + 1)}{2})^{2}$ + (k + 1)$^{3}$ \\
k$^{3}$ + (k + 1)$^{3}$ = $(\frac {k^{2}(k + 1)^{2}}{4})$ + (k + 1)$^{3}$ \\
k$^{3}$ + (k + 1)$^{3}$ = $(\frac {k^{2}(k + 1)^{2}}{4})$ + $(\frac {4(k + 1)^{3}}{4})$ \\
k$^{3}$ + (k + 1)$^{3}$ = $(\frac {k^{2}(k + 1)^{2} + 4(k + 1)^{3}}{4})$ \\ 
k$^{3}$ + (k + 1)$^{3}$ = $(\frac {(k + 1)^{2} (k^{2} + 4(k + 1))}{4})$ \\
k$^{3}$ + (k + 1)$^{3}$ = $(\frac {(k + 1)^{2} (k^{2} + 4k + 4)}{4})$ \\
k$^{3}$ + (k + 1)$^{3}$ = $(\frac {(k + 1)^{2} (k + 2)^{2}}{4})$ \\
k$^{3}$ + (k + 1)$^{3}$ = $(\frac {(k + 1) (k + 2)}{2})^{2}$, True \\
\end{enumerate}
% Q1 end

\item
\textbf{Question 2: Mathematical Induction 2} % Q2 begins
\begin{enumerate}[label=(\alph*)]
\item % 2a
P(1) = 0 = 0 \\ 
P(2) = 0 + 2 = 2 \\
P(3) = 0 + 2 + 4 = 6 \\
P(4) = 0 + 2 + 4 + 6 = 12 \\ 
\\
formula: $n^{2}$ - n \\
P(1) = $1^{2}$ - 1 = 0, P(1) = 0 \\
P(2) = $2^{2}$ - 2 = 2, P(2) = 2 \\
P(3) = $3^{2}$ - 3 = 6, P(3) = 6 \\
P(4) = $4^{2}$ - 4 = 12, P(4) = 12 \\

\item % 2b
P(n) = $n^{2}$ - n \\
P(1) = $1^{2}$ - 1 = 0, True \\
\\
Induction:
Let n = k be true: \\
P(k) = $k^{2}$ - k \\
\\
Use P(k + 1) and add (k + 1) to both sides: \\
P(k + 1) = $(k + 1)^{2}$ - (k + 1) \\
P(k + 1) = ($k^{2}$ + 2k + 1) + (-k - 1) \\
P(k + 1) = ($k^{2}$ + k), True \\
\end{enumerate}
% Q2 end

\item
\textbf{Question 3: Mathematical Induction 3} \\ \\ % Q3 begins
P(n) = n! $<$ $n^{n}$
\begin{enumerate}[label=(\alph*)]
\item % 3a
The statement P(2) states that two factorial is less than two squared. \\

\item % 3b
P(2) = 2! $<$ $2^{2}$ \\
P(2) = (2 * 1) $<$ 4 \\
P(2) = 2 $<$ 4, True \\

\item % 3c
Induction: \\
Let n = k be true: \\
P(k) = k! $<$ $k^{k}$ \\
\\
Use P(k + 1): \\
P(k + 1) = (k + 1)! $<$ $(k + 1)^{(k + 1)}$, True \\
\end {enumerate}
% Q3 end

\end{enumerate}
\end{document}
